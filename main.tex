\documentclass{article}
\usepackage[utf8x]{inputenc}
\usepackage{latexsym}
\usepackage{amsmath}
\usepackage{float}
\usepackage{graphicx}
\usepackage{booktabs}
\usepackage{url}
\usepackage{subcaption}
\usepackage{hyperref}
\usepackage{booktabs}
\usepackage{adjustbox}
\usepackage{breakcites}

\title{Gender bias in Language from a Natural Language Processing Perspective}
\author{Shawon Ashraf \\ st170090@stud.uni-stuttgart.de}
\date{29 September 2021}



\newcommand\BibTeX{B{\sc ib}\TeX}
%\addbibresource{bib.bib}

\begin{document}

\maketitle

\section*{Introduction}
Language is one of the most powerful and expressive mediums for humans to express their ideas. However it does not stop at being only a tool for communication, language also influences how humans perceive and build their thought process. This inadvertently dictates how humans develop a notion of structure and entities, for example society, class, gender etc. As Ludwig Wittgenstein had famously said that the limits of our languages define the limits of our cognitive boundary. Since language is a necessary prerequisite to describe any entity or class, which in this paper will be human gender, brings the question that if we can think of the social structure with our language and build around it, does the biases and stereotypes, specifically the ones towards gender get into language as well? Or is it that they are intertwined with human cognition and thus get carried over to the language we use for daily communication. Furthermore, for any Natural Language Processing task, language is the source of data, be it written (text corpora) or spoken (as speech signals). So this brings another question into light, if language can get biased from human cognition and the same language is used as a source for NLP applications, how much such biases affect NLP applications and our usage of language in our daily lives? In this paper, we look forward to finding answers for such queries and explore how gender bias is embedded in the ways we use our languages.

\section*{Gender in Language}
% How gender is perceived in language
To begin our discussion, we first need to look at how gender is defined in language in a formal way. As described in \cite{menegatti2017gender}, languages can be categorized \cite{braun2005cognitive} as Genderless Languages (Finnish, Turkish etc.), Natural Gender Languages (English etc.) and Grammatical Gender Languages (German, Italian etc.).

\noindent
\\
The key differentiator in this categorization is how gender is described in the lexical expressions of a language. Genderless languages do not have grammatical gender for nouns or pronouns, rather for words such as Mother, Father, Sister etc. they resort to what is described as "sex-marking". In Natural Gender Languages, pronouns are used to differentiate between genders and nouns can refer to both male and females, e.g. Doctor, Teacher, Nurse etc. Grammatical Gender Languages have their nouns distinguish between male and female entities and often, the female counterpart of a male word is inflicted from the male word. For example, in German, the female for \textit{Lehrer (Teacher)} is \textit{Lehrerin}. Again, the gender of a word is often defined by the surrounding parts of speech, e.g. \textit{Löffel(Spoon)} is masculine and \textit{Gabeln(Fork)} is feminine. To keep our discussion on point we are going to focus solely on the natural use of gender in language with sex-marking instead of grammatically motivated gender since it corresponds to the topic at hand.

\section*{Social View of Gender}
According to \cite{larson2017gender}, gender can be perceived as a social construct in the following ways : 


\begin{figure}[H]
    \centering
    \includegraphics[width=0.8\textwidth]{Gender Social View.png}
    \caption{Gender as a social construct}
    \label{fig:gender_social_view}
\end{figure}

\noindent
The folk refers to common beliefs regarding gender orientation in a society. These views can be motivated by age old traditions or religion or both. The folk point of view regarding gender can be limiting depending on the culture and other factors and also a big contributor to our topic discussion which is bias towards a specific gender. We will discuss this effect in a later section of the paper. \\


\noindent
The performative view towards gender is based on what roles society assigns to a gender and expects people to conform to. These roles can be motivated by the folk point of view but the society can also have its own. For example in matriarchal societies in many Asian countries, women are regarded as the key figures and they have the final say when it comes to taking decisions. Likewise in a different form of society, the norms will expect that some tasks or occupation to be handles only by the male members while their female counterparts should maintain the households. We can also say that this a more constrained and imposed view of gender. \\

\noindent
The psychological view on the other hand is not constrained by society or traditional norms. This view rather focuses on the characteristics(such as self-reliance, independence, loyalty, sympathy etc.) of the existing gender and tries to find a cluster based on these characteristics. \\

\noindent
However, \cite{larson2017gender} also suggest that devising a hierarchy like the one described here is an open ended task and is much more dependent on the research question on what someone expects to do with gender. 

\section*{Gender Bias in Linguistic Processes}
We have seen how gender in language can be motivated by the linguistic structure as well as social structures in the previous sections. The question is, how do they connect to each other? From a neutral perspective, linguistic or grammatical role for words should not effect how they are imagined in a social hierarchy, however the reality is the opposite. \\

\subsection*{Gender stereotypes}
The first case of gender bias is traced in verbal communication and the roles of communicators in it \cite{menegatti2017gender}. Here, the problem is considered two fold, one being gender stereotypes, which expects gender specific traits or choice of words for communication, the second being considering male form as the general prototype of a human being in many languages. \\ 

\noindent
Gender stereotypes expect communal or softness or warmth based traits from females and agentic or more competitive traits from males \cite{cuddy2008warmth}. For example, a common gender stereotype would be to compare women with flowers while comparing men with wars and weapons. Such stereotypes lead to role related words being tied to a specific gender. In turn, this leads to giving male roles more power and status, as such roles of higher stature have mostly been held by men in history. This is the social effect, when we connect it to language, we can see that systematically marking some roles for a specific gender creates a cognitive conception that only the gender in contention is suitable for such roles and not the other, thus introducing gender discrimination in the process. This discrimination translates to language when these roles get described in language and over the time get integrated in the structure of the language.

\subsection*{Linguistic Conventions and Abstractions}
Linguistic conventions on the other hand tend to be gender neutral towards treating words. Despite that, existing gender stereotypes make gender bias normative and over time, the same linguistic conventions can no longer remain neutral. Again, linguistic abstraction becomes another way to introduce gender bias \cite{ng2007language}. There can be words with multiple meanings based on the context and when the users of a language tend to use such words in a specific context, they subconsciously bring forth their gender stereotypes \cite{rubini2014strategic}. \\ 

\noindent
This entire process becomes a cycle of social roles for genders transforming to gender stereotypes and then eventually making their way into daily use of a language. We can also assume here that gender bias is language has no single contributor, rather it is influenced systematically, as language, society, human cognition are all intertwined. 

\subsection*{Male as a general prototype in languages}
According to \cite{eagly1994people}, communal roles for women are more to assert male dominance in the society and attribute women more as a subordinate to men. This inadvertently creates a stereotype that men should be prioritised over women to describe a human prototype. This falls in line with ancient scriptures and texts depicting men as leaders, warriors and intellectuals mostly, while women in history fade into oblivion. \\ 

\noindent
Furthermore, this attitude influences the lexicon of a language as well. For example, words or entities such as \textit{Virgin}, \textit{Working Mother}, \textit{Career Women} etc. have no male counterparts \cite{budziszewska2014men} \cite{maass1996language}.

\subsubsection*{Masculine Generics}
To extend the topic of perceiving male as a general or generic prototype, we need to look at masculine generics and how they favor males over females. Generic forms for gender comprise mostly of masculine words whereas feminine words are only used to describe female entities. Frequent usage of masculine forms as generics validate the subconscious  gender stereotypes in language users. This can also be associated with the term \textbf{linguistic normativity effect} \cite{menegatti2017gender}, where people tend to associate general forms of a language to a gender which they think should be prioritized. As such, masculine generics, due to their nature of usage, do not consider men and women to be equal and due to women being less relevant to generics, they fail to contribute as a generic entity. Studies from \cite{braun2005cognitive}, \cite{sczesny2016can} have shown that when people are asked to assign a gender to some entity they do not know about, they tend to choose masculine gender more over feminine. When applied to a professional setting, e.g. \textit{Sports}, this stereotype can lead to the notion that in a specific occupation or field of work, women are always outnumbered by men \cite{stahlberg2007representation}. 


\section*{Bias in NLP}
Source of bias in data, how it's connected to cognitive and linguistic proponents stated above

\section{Gender in NLP}
Ethical issues, should gender be used as a variable at all, various nlp tasks and bias in them (summaries and opinions on papers)

\section{Mitigating Bias}
NLP and regular language

debiasing in nlp
gender fair expressions
social change

\section{Conclusion}
Sum up alles

\clearpage
\bibliography{bib.bib}
\bibliographystyle{acl_natbib}

\end{document}
